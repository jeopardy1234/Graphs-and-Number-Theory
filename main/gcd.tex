\chapter{Greatest Common Divisor}

\section{Introduction}
Greatest Common Divisor or GCD of two numbers $a$ and $b$ is the largest number $x$, such that it divides both $a$ and $b$. For example, the GCD of $32$ and $20$ is $4$. This is because $4$ is the largest number which divides both $32$ and $20$. In this chapter, we will first discuss Euclidean algorithm to find GCD. Then, we will look at the Extended Euclidean algorithm. This will form the basis for upcomping chapters.

\section{Euclidean Algorithm for GCD}
The algorithm is fairly simple. It states that
\begin{equation}
    \text{gcd}(a,b)=
    \begin{cases}
      a, & \text{if}\ b=0 \\
      \text{gcd}(b,a\%b), & \text{otherwise}
    \end{cases}
\end{equation}

\subsection{Proof of Euclidean Algorithm}
Proving the fact that $gcd(x,y) = gcd(x-y,y)$ would suffice
  This is true because:

\begin{enumerate}
   \item Any integer that divides both $x$ and $y$ also divides $x-y$, so $gcd(x,y) \le gcd(x-y,y)$.  
   \item Any integer that divides both $x-y$ and $y$ also divides $x$ and $y$. So, $gcd(x,y) \ge gcd(x-y,y)$
\end{enumerate}
Hence, $gcd(x,y) = gcd(x-y,y)$
\subsection{Time Complexity}
Notice that if $a \ge b$ then $a\%b < \frac{a}{2}$
\subsubsection{Proof:}
Lets take two cases:\\
\begin{enumerate}
  \item $b \le \frac{a}{2}$: This is obvious
  \item $b > \frac{a}{2}$: $a\%b \le a-b$ and $a-b < \frac{a}{2}$ 
\end{enumerate}
Hence, the time complexity is $O(\log_2(\text{min} (a,b)))$\\
The code is quite simple:

\begin{minted}
[
frame=lines,
framesep=2mm,
baselinestretch=1.2,
fontsize=\footnotesize,
linenos
]
{cpp}
int gcd(int a, int b)
{
	if(b == 0)
		return a;
	return gcd(b, a%b);
}
\end{minted}

\section{Extended Euclidean Algorithm}
There's a theorem which goes as follows: 
If $d$ divides both $a$ and $b$ and $d = ax+by$ for some integers x and y, then necessarily $d = \text{gcd}(a,b)$. We are not really interested in proving this theorem. What we care about some values of $x$ and $y$ which satisfy this equation. \\
Let's denote $gcd(a,b)$ as $g$. We will take help of Euclid's algorithm to solve this problem.\\
Since $gcd(b, a\%b) = gcd(a,b) = g$,\\
\begin{equation}
\begin{split}
ax + by &= g \\
bx_1 + (a\%b)y_1 &= g
\end{split}
\end{equation}
\\
Now, $a\%b = a-\floor*{\frac{a}{b}}b$\\
Substituting this in $(6.2)$ we get

\begin{equation}
\begin{split}
x &= y_1\\
y &= x_1 - y_1 \floor*{\frac{a}{b}}
\end{split}
\end{equation}

That's it! We keep using this recursively. The code would look like this:

\begin{minted}
[
frame=lines,
framesep=2mm,
baselinestretch=1.2,
fontsize=\footnotesize,
linenos
]
{cpp}
pair<int,int> extended_euclid(int a, int b)
{
    if(b == 0)
    {
        return {1,0};
    }
	int x1, y1;
    tie(x1,y1) = extended_euclid(b, a%b);

    int x = y1;
    int y = x1 - (a/b)*y1;

    return {x,y};
}
\end{minted}